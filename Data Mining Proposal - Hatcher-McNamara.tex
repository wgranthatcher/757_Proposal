% WILLIAM GRANT HATCHER, KEVIN MCNAMARA --  COSC 757 -- Project Proposal 

% Due date: Oct. 4, 2018 

%TOPIC: Data Mining

% Sections: 

%	Motivation
%		-Exec Summary
%	Problem Statement
%		-Problem
%		-Objectives (Motivation)
%	Working plan
%		-Approach and Management Plan
%		-Schedule
%		-Qualifications
%	Evaluation plan
%		-Success Metrics
 


\documentclass[12pt]{article}
\usepackage[letterpaper, margin=1in]{geometry}
\usepackage{enumerate}
\usepackage{titlesec}
\usepackage{color}
\titlelabel{\thetitle. }

\usepackage{geometry}
\geometry{verbose,tmargin=1in,bmargin=1in,lmargin=1in,rmargin=1in}

\begin{document}



%  -----   COVER PAGE    ------
\pagenumbering{roman}
\pagenumbering{gobble}

\begin{center}
	\Large{\bfseries Proposal }\\[18pt]
\end{center}

\begin{flushleft}
	%\item{\bfseries Funding Opportunity:}  Secure and Trustworthy Cyberspace  (SaTC)\\[11pt]
	%\item{\bfseries Funding Opportunity Number:} 17-576 \\[11pt]
	%\item{\bfseries Name of Offeror/Institute:} Wei Yu --- PI, Towson University\\[11pt]

	\begin{center}
		\Large Data Mining and Predictive Modeling of Amazon Customer Reviews\\[30pt]
	\end{center}

	%\item{\bfseries Designation:} EDU \\[11pt]

	\item{\bfseries Technical Points of Contact:}

		\hspace{4ex} Name: William G. Hatcher
		
		\hspace{4ex} Email Address: whatch2@students.towson.edu
		
		\hspace{4ex} Mailing Address: Department of Computer and Information Sciences
		
		\hspace{22ex} Towson University, 7800 York Road, Towson, MD 21252
		\\[11pt]
 
 
		 \hspace{4ex} Name: Kevin McNamara
		 
		 \hspace{4ex} Email Address: kmcnamara@towson.edu
		 
		 \hspace{4ex} Mailing Address: Department of Marketing
		 
		 \hspace{22ex} Towson University, Stephens Hall, Towson, MD 21252
		 \\[11pt]
 
	%\item{\bfseries Type of Legal Instrument:} Grant\\[11pt]
	%\item{\bfseries Funding Requested:} \$100,000.00\\[11pt]

	\item{\bfseries Period of Performance:} 10/4/2018 - 12/6/2018 (2 months)\\[48pt]

	\begin{center}
		\Large{\bfseries Executive Summary }\\[11pt]
	\end{center}

\end{flushleft}

With 43.5% market share in 2017, Amazon is the leader in online retailing in the US. The tech giant possesses a wealth of data regarding customer purchases, preferences, reviews, and more. Using data mining techniques, we can harness this data to research patterns in customer sentiments and buying habits. In this proposal, we plan on exploring a set of Amazon customer reviews to analyze patterns between various attributes and develop a predictive tool for discovered correlated variables. \\[2pt]


%  -----   Table of Contents    ------
\newpage\tableofcontents


%  -----   Body    ------
\newpage
\pagenumbering{arabic}
\clearpage
\setcounter{page}{1}

\section{Introduction}

	\subsection{Problem Statement}\hspace{4ex} Current dataset contains over 9,000,000 individual Amazon product reviews containing attributes such as Customer ID, Product ID, Star Rating, Helpful Votes, Total Votes, Helpful Review (Y or N) and review date. The dataset, while detailed, does not relate reviews to each other or correlate attributes to show patterns.   
	
	
	\subsection{Challenges}\hspace{4ex} Deep learning now suffuses the applications and software products that make daily life easier and allow us to stay connected. Yet, it may be impossible to cover every area that deep neural networks have been applied. In a research context, looking only at the IEEE Xplore digital library, some 7,593 articles have been published relevant to deep learning. In addition, enterprise and commercial deep learning systems can only be assessed from the corporate and trade publications they deign to release, often seeking to keep hidden the mechanisms that may indeed be trade secrets. Thus, to properly survey the landscape of deep learning, it is imperative to have a thorough understanding of deep learning architectures, and to be able to infer and discriminate what information is truly novel and unique. \\
		
	\subsection{Importance}\hspace{4ex} Deep learning is at a critical peak of public awareness. As more researchers are investigating the application of deep learning to their field or topics of study, it is necessary for those without the knowledge to have a reference to assist them in applying appropriate deep learning approaches, whether they be classification, regression, data fusion, reinforcement learning, etc. In addition, for the general Computer Science community, and those that work daily with deep learning architectures, the breadth and depth of deep learning holds many novel and interesting works that may perhaps provide inside into solving a particular problem. 
	
	\subsection{Objectives}\hspace{4ex}In this project, we plan to evaluate the state-of-the-art of Deep Learning across platforms, algorithms, and datasets. Specifically, we intend to categorize Deep Learning by mechanism, algorithm, hyperparamters, platform, applications, and performance. This includes developing a survey of the works done in this area, and conducting benchmark tests to provide baseline comparative analysis. 

\section{Project Approach and Management Plan}

	\subsection{Overview}\hspace{4ex}Two primary deliverables are intended in this work. First, a survey of deep learning shall be conducted, considering the origins and recent advances in deep neural networks. This survey shall categorize deep learning architectures into various categories and subcategories derived from the type of learning, learning target, algorithmic implementation, enabling software frameworks, and datasets. Then, we shall carry out a practical evaluation of many of the representative software platforms to investigate their efficiency, diversity, and ease of use. 
	
	\begin{flushleft}
		\item{\textbf{Deliverables:}}
		
		\hspace{4ex} a. \textit{Deep Learning Survey:} The survey will consider the evolution of Machine Learning to the current Deep Learning paradigm, the enabling algorithms and technologies, a classification of the available platforms,  current and emerging applications of Deep Learning, and future research directions. The survey will review cross-sections of supervised learning, reinforcement learning, and unsupervised learning, and review the advances in convolutional neural networks, deep belief networks, deep Q-learning, and more.
		
		\hspace{4ex} b. \textit{Practical Implementation and Evaluation:} The practical evaluation of Deep Learning platforms will test the comparative performance of the various platforms and their comparable algorithms in terms of accuracy and runtime on various datasets. Each dataset represents a different learning application (classification, anomaly detection, etc.). Platforms include TensorFlow, Theano, Keras, Torch, Deeplearning4J, among others. \\[11pt]
		
		\item{\textbf{Datasets:}}
		
		\hspace{4ex} a. \textit{Handwriting Samples:} The MNIST dataset is a widely used dataset for machine learning of handwritten numbers. This dataset is provided the National Institute of Standards and Technology, and comprises some 60,000 samples. 
		
		\hspace{4ex} b. \textit{Image Classification:} The ImageNet dataset, composed of over 14 million labeled images, is another widely used dataset, and was part of the ImageNet competition. This dataset is used for object and scene recognition.
		
		\hspace{4ex} c. \textit{Speech Recognition:} The TIMIT dataset includes recordings of 630 speakers of eight major American English dialects, each reading ten sentences.  
		
		\hspace{4ex} d. \textit{Text Data:} The Examiner.com crowd-sourced news website data includes over 3 million instances of headlines, spam, etc. for clustering and sentiment analysis.
		
		\hspace{4ex} e. \textit{Mobile Malware:} In the course of our prior work, we have assembles some 60,000 Android Applications (Benign and Malware) from various sources for use in android malware detection.  
		
		\hspace{4ex} f. \textit{Network Traffic:} Multiple datasets have been requested from the Center for Applied Internet Data Analysis (CAIDA), which includes public and upon-request anonymized network traces from 2008-2016, as well as various worm trace snapshots.
		
	\end{flushleft} 
	
	\subsection{Schedule}\hspace{4ex}Based on the overview provided, two major tasks are to be delivered. The first, the survey of deep learning, shall be conducted in the first three months of the period of performance, and shall be delivered by {\em March 31, 2018}. 
	
	The second deliverable, the comparative analysis of deep learning platforms, shall be carried out over the remaining three months to be delivered at the completion of the performance period, on {\em June 30, 2018}. This will entail the evaluation of at least four major deep learning platforms on the six datasets noted in the \textbf{Dataset} subsection of Section 2.1. 
	
	\subsection{Qualifications}\hspace{4ex}This research group has conducted many successful surveys in the past, and has been conducting research on machine learning and deep neural networks for some time. Publications of this work include various IEEE conferences and journals. The most recent works include deep learning for Android malware detection, using the malware dataset outlined above. Other publications include network traffic analysis using distributed parallel systems, among others. We feel this work has a high chance of success, given the current demand for the topic, and the relevancy of the review of deep learning platforms. Specifically, such platforms abound, and navigating the various aspects of deep learning implementations requires no small amount of experience. In addition to these qualifications, we have access to multiple server-grade computer systems in our research lab, as well as access to the IEEE Xplore digital library for reference.   

\section{Evaluation of Success}\

The success criteria of this projects are based on two primary metrics: (i) the timely delivery of the deliverables by the proposed due dates (Merch 31 and June 30 of 2018), and (ii) the acceptance and publication of both works. In the case of publication, the survey must be accepted by a reputable journal, and not simply a conference, while the comparative assessment will still be considered successful only being accepted for conference publication.

\end{document}
